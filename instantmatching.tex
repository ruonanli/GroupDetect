% !TEX root =  GroupDet.tex
\section{Matching and Localization of Interactions}

\vspace{-5pt}

\subsection{Representation}

\vspace{-5pt}

We consider a video as a sequence of $T$ temporal units that occur at a frequency equal to or less than the frame-rate of the raw video data. The duration of these $T$ units is typically between one and a  few raw video frames, and it is determined by one's application-appropriate choice for temporal resolution of computing atomic action descriptors (e.g., positions, velocities, accelerations, histograms of flow, space-time SIFT). We assume the existence of an application-specific detection and tracking system that outputs $M$ space-time tracks, which can be time-varying points, bounding boxes, silhouettes, or something else. Due to agent entry and exit, occlusions, and other tracking errors, not all $M$ tracks will persist over all $T$ frames, and some of the $M$ tracks may correspond to short-lived false detections.  The value of $M$ is thus the total number of trajectory fragments that are identified with distinct agents.

With each track we associate ensembles of two types of descriptors. There are $TM$ per-time-unit $d_{I}$-dimensional descriptors $\{\mathbf{f}_{m,t}\}$ where $\mathbf{f}_{m,t}$ encodes the $m$th agent's activity at time unit $t\in[1, T]$; and $TM(M-1)$ pairwise $d_{P}$-dimensional descriptors $\{\mathbf{g}_{m,m',t}\}$ where $\mathbf{g}_{m,m',t}$ encodes at time $t$ the motion and/or appearance of agent $m$ relative to agent $m', m'\ne m$. Loosely speaking, $\mathbf{g}_{m,m',t}$ captures the ``influence'' that agent $m'$ has over agent $m$ at time $t$. This influence is not symmetric in general, so typically $\mathbf{g}_{m,m',t}\ne \mathbf{g}_{m',m,t}$.  We use the notation $\mathcal{Q}_{t}\triangleq\{\mathbf{f}_{m,t},\mathbf{g}_{m,m',t}\}$ for the ensembles of all $M$ tracks at time $t$, and $\mathcal{Q}\triangleq\{\mathcal{Q}_{t}\}_{1\leq t\leq T}$ for the ensembles harvested from the entire input video. As mentioned above, the dimensions and entries in the descriptor vectors $\mathbf{f}$, $\mathbf{g}$ will be application dependent, and we consider a variety of examples in our experiments.

Each exemplar video is processed in the very same way as the input video, so that an exemplar of $N\le M$ participants over $S$ time units is represented at each time $s\in [1,S]$ by the ensemble $\mathcal{D}_{s}\triangleq\{\mathbf{f}^{D}_{n,s},\mathbf{g}^{D}_{n,n',s}\}$. We use the analogous notation $\mathcal{D}\triangleq\{\mathcal{D}_{s}\}_{1\leq s\leq S}$ for the ensembles collected from the entire exemplar. 

\vspace{-5pt}
\subsection{Matching between Temporal Units}

\label{agg}
\vspace{-5pt}

The first step in our framework is to separately compute the correspondence between the $N$ exemplar agents at each time $s\in[1, S]$ and the optimal subset of $N\le M$ of input agents at each time $t\in[1, T]$. We represent this $N$-to-$M$ correspondence by the $N\times M$ binary matrix $W$, where the $nm$-th entry $w_{nm}$ is one only when the $n$th exemplar agent is matched to the $m$th input agent.\footnote{Missing input tracks at time $t$ are handled by setting the values of the missing descriptors to be sufficiently large (or small) so as not to be matched with any agents in the exemplar.} Matches must be unique, so these matrices must have one non-zero entry in each row and at most one non-zero entry in each column: $W\mathbf{1}=\mathbf{1}$ and $\mathbf{1}^{T}W\leq\mathbf{1}^{T}$. We use the symbol $\mathcal{W}$ to represent the space of all such matrices, \ie, $\mathcal{W}\triangleq\{W\in\{0,1\}^{N\times M}| W\mathbf{1}=\mathbf{1}, \mathbf{1}^{T}W\leq\mathbf{1}^{T}\}$.

The quality of a correspondence is measured by the similarity between the individual and pairwise descriptors of the $N$ selected input agents and those of the $N$ exemplar agents. We formalize this by defining 
\begin{equation}
\vspace{-10pt}
\begin{split}
&\hat{D}(\mathcal{Q}_{t}, \mathcal{D}_{s}, W)=\\
&\sum_{nm}w_{nm}d_{I}(\mathbf{f}_{m,t}, \mathbf{f}^{D}_{n,s})+\!\!\!\!\!\!\sum_{nmn'm'}\!\!\!\!\!w_{nm}w_{n'm'}d_{P}(\mathbf{g}_{m,m',s}, \mathbf{g}^{D}_{n,n',t}),
\end{split}
\vspace{-10pt}
\end{equation}
to be the dissimilarity between two instantaneous ensembles under a particular matching matrix $W$. We use Mahalonobis distances to compare descriptors in this expression, so that $d_{I}(\mathbf{f}, \mathbf{f}')=(\mathbf{f}-\mathbf{f}')^{T}\Sigma_{I}(\mathbf{f}-\mathbf{f}')$ and $d_{P}(\mathbf{g}, \mathbf{g}')=(\mathbf{g}-\mathbf{g}')^{T}\Sigma_{P}(\mathbf{g}-\mathbf{g}')$, with $\Sigma_{I}\succeq 0$ and $\Sigma_{P}\succeq 0$ positive semi-definite matrices learned from exemplar videos as will be described in Sec.~\ref{MetLearn}.

Our immediate objective is to find the matching matrix $W\in\mathcal{W}$ that minimizes the score $\hat{D}(\mathcal{Q}_{t}, \mathcal{D}_{s}, W)$. Letting $\mathbf{w}$ be the vector formed by stacking the columns of $W$, the optimization can be expressed as
\begin{equation}
\label{Prob1}
\min_{w_{nm}\in\{0,1\}}\mathbf{c}^{T}\mathbf{w}+\mathbf{w}^{T}H\mathbf{w},\ \ \textup{s.t.}\ W\mathbf{1}=\mathbf{1}, \mathbf{1}^{T}W\leq\mathbf{1}^{T},
\end{equation}
where $\mathbf{c}$ is a $MN\times 1$ vector of distances between individual descriptors, $d_{I}(\mathbf{f}_{m,t}, \mathbf{f}^{D}_{n,s})$, and $H$ is a $MN\times MN$ matrix of distances between pairwise descriptors $d_{P}(\mathbf{g}_{m,m',t}, \mathbf{g}^{D}_{n,n',s})$'s. This problem has integer constraints and is generally not convex, so we instead solve
\begin{equation}
\label{Prob2}
\min_{w_{nm}\in[0,1]}(\mathbf{c}+\hat{\mathbf{c}})^{T}\mathbf{w}+\mathbf{w}^{T}(H+\hat{H})\mathbf{w}, \textup{s.t.} W\mathbf{1}=\mathbf{1}, \mathbf{1}^{T}W\leq\mathbf{1}^{T},
\end{equation}
where $\hat{\mathbf{c}}=[\sigma_{1},\sigma_{2},\cdots,\sigma_{MN}]^{T}$, $\hat{H}=\text{diag}\{-\sigma_{1},-\sigma_{2},\cdots,-\sigma_{MN}\}$, and each $\sigma_{i}$ is a sufficiently large number greater than $\sum^{MN}_{j=1,j\neq i}|H_{ij}|+H_{ii}$. 

Note that $\hat{H}$ imposes a negative strictly dominant diagonal to $H$ and the quadratic term $\hat{H}+H$ is strictly negative definite. Therefore, (\ref{Prob2}) is a concave programming in the convex unit hypercube $[0,1]^{N\times M}$ and will achieve its minimum at one of the feasible vertices. The feasible vertices, meanwhile, are exactly the feasible solutions of (\ref{Prob1}), and at these vertices, the values of the objective of (\ref{Prob2}) are equal to those of (\ref{Prob1}) due to the cancellation brought by $\hat{\mathbf{c}}$. It is therefore implied that by solving the much more efficient Problem (\ref{Prob2}) we obtain the exact solution for the original Problem (\ref{Prob1}). We solve (\ref{Prob2}) using the CVX toolbox \cite{cvx}.

For notational convenience in later sections, we define $D(\mathcal{Q}_{t}, \mathcal{D}_{s})\triangleq\min_{W\in\mathcal{W}}\hat{D}(\mathcal{Q}_{t}, \mathcal{D}_{s}, W)$ to be  the similarity between ensembles $\mathcal{Q}_{t}$ and $\mathcal{D}_{s}$, and $W^{t,s}\triangleq\arg\min_{W\in\mathcal{W}}\hat{D}(\mathcal{Q}_{t}, \mathcal{D}_{s}, W)$ to be the optimal instantaneous matching matrix that yields this similarity.

