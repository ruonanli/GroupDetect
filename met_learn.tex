\section{Large-Margin Ensemble Dissimilarity}
\label{MetLearn}


Recall that in Section \ref{vote}, we expect the score $D$ to be small when a temporal unit in the input is best aligned to a temporal unit in the exemplar by the `correct' matching matrix. In Section \ref{BB}, we expect the dissimilarity $D^{*}(t)$ are driven toward zero in the `ground-truth' interval $[t_{s}, t_{e}]$ , and toward a large positive number $2\Delta$ otherwise. For both purposes, in this section we learn an effective ensemble dissimilarity measure using the annotated database. Recall that the ensemble dissimilarity $D$ or $\hat{D}$ is aggregated from `atomic' descriptor distances $\{ d_{I}(\mathbf{f}_{m,t}, \mathbf{f}^{D}_{n,s}), d_{P}(\mathbf{g}_{m,m',t}, \mathbf{g}^{D}_{n,n',s})\}$, and the ensemble dissimilarity learning therefore narrows down to learning effective `atomic'  distances.  We parameterize an atomic distance function using a Mahalanobis distance metric, \textit{i.e.}, let $d_{I}(\mathbf{f}, \mathbf{f}')=(\mathbf{f}-\mathbf{f}')^{T}\Sigma_{I}(\mathbf{f}-\mathbf{f}')$ be the atomic distances between two individual activity descriptors, and $d_{P}(\mathbf{g}, \mathbf{g}')=(\mathbf{g}-\mathbf{g}')^{T}\Sigma_{P}(\mathbf{g}-\mathbf{g}')$ be that between two contextual descriptors, where $\Sigma_{I}\succeq 0$ and $\Sigma_{P}\succeq 0$  are positive semi-definite matrices.  

We learn each pair of Mahalanobis metrics $(\Sigma_{I}, \Sigma_{P})$ for each number of participants separately. To do so, we construct two different types of collections from our database exemplars. The first collection, $\mathcal{P}$, contains all pairs of instantaneous interaction ensembles that are similar to each other. The second collection, $\mathcal{M}$, is comprised of all ordered triples $(h,k,l)$ in which ensemble $h$ is similar to ensemble $k$ but they two  are dissimilar to ensemble $l$. In the present case, we build the collection $\mathcal{P}$ and the ($(h,k)$-indexed) similar pairs in collection $\mathcal{M}$ using all instantaneous ensemble pairs with their ground-truth matching matrices from the same category extracted from the same cell at the lowest level of the pyramid. Similarly, we include in the ($(h,l)$-indexed and $(k,l)$-indexed) dissimilar pairs  of collection $\mathcal{M}$ the ensembles extracted from: 1) different pyramid cells of same-category exemplars; 2) different-category exemplars in the same cell; and 3) same-category same-cell ensembles with simulated wrong matching matrices. The `background' exemplars describing non-interaction intervals are included in $\mathcal{M}$ as well. Having defined the collections $\mathcal{P}$ and $\mathcal{M}$, we find the Mahalanobis metrics by solving
\begin{equation}
\label{classify}
\begin{split}
&\min_{\Sigma_{I},\Sigma_{P}} \sum_{(u,v)\in\mathcal{P}}\hat{D}(\mathcal{D}_{u}, \mathcal{D}_{v}, W_{u,v})+\gamma\sum_{(h,k,l)\in\mathcal{M}}\xi_{h,k,l}\\
&\textup{s.t.}  \hat{D}(\mathcal{D}_{h}, \mathcal{D}_{l}, W_{h,l})-\hat{D}(\mathcal{D}_{h}, \mathcal{D}_{k}, W_{h,k})\ge 2\Delta-\xi_{h,k,l}, \\
&\xi_{h,k,l}\ge0, \Sigma_{I}\succeq 0, \Sigma_{P}\succeq 0, \forall (h,k,l)\in\mathcal{M}.
\end{split}
\end{equation}
Note that $\Delta$ is the large positive number as used in (\ref{quality}), and (\ref{classify}) will encourage a similar pair to contribute a negative summand around $-\Delta$ to (\ref{quality}), and encourage a dissimilar pair to contribute approximately $+\Delta$ summand,  thus enabling the branch-and-bound search as introduced in Section \ref{BB}.  The minimization over either $\Sigma_{I}$ or $\Sigma_{P}$ falls into the framework of large margin nearest neighbor (LMNN) formulation \cite{Weinberger:ML}. Consequently, we simply decompose problem (\ref{classify}) into independent LMNN tasks and employ \cite{Weinberger:ML} for each of them.

A recap of the entire approach presented above distinguish our work from similar ones \cite{Ryoo:group,Amer:group}: \cite{Ryoo:group} does not identify the participants, and relies on a set of pre-determined application-specific rules for temporal detection. \cite{Amer:group} builds a probabilistic generative model for a set of histograms of pose-coded detections computed at a dense spatio-temporal grid and implicitly infers participants using these `hidden' histograms, while we adopt a data-driven bottom-up strategy. Though parametric generative model such as used in \cite{Ryoo:group} is powerful in explaining human motion governed by a structured underlying dynamic, our approach is more generic and flexible in addressing either structured or less-structured dynamics such as those in our classroom video collection. We achieve quite reliable detection and tracking in our experiments, and our approach to directly match these tracks of targets turns out effective as to be demonstrated in Section \ref{expall}. As the low-level detection and tracking algorithms continue to improve, we expect our framework to become more applicable as well.

